\chapter{Fundamentos do funcionamento e infraestrutura da Internet}
\begin{minipage}[t][10\baselineskip][t]{\linewidth}
\begin{epigraph}[Profiles Of The Future: An Inquiry into the Limits of the Possible, 1958]{Arthur C. Clarke}
Any sufficiently advanced technology is indistinguishable from magic.
%\begin{epigraph}[Nineteen Eighty-Four, 1949]{George Orwell}
%he who controls the past controls the future, and he who controls the present controls the past
\end{epigraph}
\end{minipage}

Devido � sua organiza��o descentralizada, a Internet � uma plataforma de comunica��o mais robusta a desastres do que os seus antecessores telefonia, r�dio e TV. Para um desses servi�os ficarem indispon�veis, basta que se destrua sua torre de transmiss�o, no caso de r�dio e TV, ou central de comuta��o no caso dos telefones. Essa maior robustez da Internet, entre outras novidades, acabou gerando uma propaganda em seu favor, no in�cio da sua populariza��o, que chegou ao ponto de classific�-la erroneamente como incensur�vel em \cite{time-dewitt}. Para o leitor ter uma vis�o mais cr�tica sobre o que se veicula nos meios de comunica��o e entender melhor o que ser� apresentado neste trabalho, este cap�tulo ir� resumir os conceitos b�sicos que tangem o funcionamento da Internet e sua infraestrutura.

\section{Se��o 1}

\subsection{Subse��o a}
\lipsum[1]

\subsubsection{Subsubse��o I}
\lipsum[1]

\subsection{Subse��o b}
\lipsum[1]

