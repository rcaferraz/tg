\chapter{Vigil�ncia na Internet}
%\begin{minipage}[t][10\baselineskip][t]{\linewidth}
%\begin{epigraph}[Profiles Of The Future: An Inquiry into the Limits of the Possible, 1958]{Arthur C. Clarke}
%Any sufficiently advanced technology is indistinguishable from magic.
%\end{epigraph}
%\end{minipage}

Conforme diz \cite{adams06} e \cite{tor-design} Introdu��o. Lorem ipsum dolor sit amet, consectetur adipiscing elit. Nunc placerat vulputate auctor. Praesent ullamcorper sem lectus, non molestie ipsum mollis sed. Nunc pharetra, est vel sodales sodales, nunc massa sodales dui, eget tristique odio nisi at risus. Suspendisse tempus magna orci, et mollis neque commodo quis. Fusce vel enim mauris. Etiam vitae nisi a nisl sagittis volutpat. Nunc pretium ligula eu convallis scelerisque. Nam laoreet arcu vel erat interdum, ac vestibulum mauris ullamcorper.

\section{Transmiss�o volunt�ria dos dados}

\subsection{Infec��o do computador}

Explicar sobre Cavalo de Troia. � poss�vel ficar escutando o que entra/sai do computador pela rede, para tentar encontrar algum tr�fego suspeito, possivelmente gerado por um Cavalo de Troia.

\subsection{Infection Proxies}

Palestra "Bugged Planet" de Andy M�ller fala sobre Infection Proxies.

\section{Intercepta��o dos dados}

Mesmo com a m�quina limpa, o usu�rio pode ser vigiado � dist�ncia. Neste caso o usu�rio n�o tem como investigar se est� sendo vigiado. A intercepta��o pode ser feita em v�rios lugares: intercepta��o de sinal WiFi, ou instala��o de escutas no ISP a quem voc� se conecta, ou em um IXP que o seu ISP utiliza, no host com quem voc� se comunica,

\section{Com quem voc� se comunica}

� poss�vel descobrir n�o os dados, mas o que voc� acessa, atrav�s da descoberta das suas requisi��es a algum servidor DNS.

\lipsum[1]

