% Author: Rodolfo Ferraz.

% Build this file with pdflatex 01_Intro.tex
\documentclass[xcolor=dvipsnames]{beamer}
\usecolortheme[named=Brown]{structure} 
%\usetheme{Rochester}
\usetheme{Frankfurt}

\usepackage{booktabs}
\usepackage{graphics}
\usepackage{multirow}
\usepackage[latin1]{inputenc}   % para os acentos
\usepackage[brazil]{babel}      % para hifeniza\c{c}\~{a}o
\usepackage{color}

%\usepackage[english]{babel}
%\usepackage[utf8]{inputenc}

\definecolor{lightgrey}{rgb}{0.8, 0.8, 0.8}
\newcommand{\shd}[1]{\colorbox{lightgrey}{#1}}

\newcommand{\tblue}[1]{\textcolor{blue}{#1}}
\newcommand{\tred}[1]{\textcolor{red}{#1}}
\newcommand{\tgreen}[1]{\textcolor{green}{#1}}
\newcommand{\tyellow}[1]{\textcolor{yellow}{#1}}


\newcommand{\byellow}[1]{\colorbox{yellow}{#1}}
\newcommand{\bred}[1]{\colorbox{red}{\textcolor{white}{#1}}}
\newcommand{\bgreen}[1]{\colorbox{green}{#1}}

\definecolor{lightestgrey}{rgb}{0.91, 0.91, 0.91}
\newcommand{\lshd}[1]{\colorbox{lightestgrey}{#1}}

\title{M�todos governamentais de censura e vigil�ncia na Internet}
\author{Rodolfo Cesar de Avelar Ferraz}
\institute{Centro de Inform�tica -- Universidade Federal de Pernambuco}

%\date[PAAP 2011]{Fourth International Symposium on Parallel Architectures, Algorithms and Programming}
%\pgfdeclareimage[height=0.4cm]{cc}{license}
%\pgfdeclareimage[height=1.0cm]{logo}{ines}
%\pgfdeclareimage[height=0.8cm]{logo2}{spg}
\pgfdeclareimage[height=0.8cm]{logo3}{cin}

%\logo{\pgfuseimage{logo3}\hspace{0.1cm}\pgfuseimage{logo}\hspace{0.1cm}\pgfuseimage{logo2} }
\logo{\pgfuseimage{logo3}}

\begin{document}

\frame{\titlepage}

\begin{frame}
	\frametitle{Agenda}
	\begin{itemize}
	    \item {\bf Motiva��o}
	    \item {\bf Fundamentos}
	    \item {\bf Censura na Internet}
	    \item {\bf Vigil�ncia na Internet}
	    \item {\bf M�todos para contornar vigil�ncia e censura na Internet}
	    \item {\bf Considera��es Finais}
	\end{itemize}

\end{frame}

\section{Motiva��o}
\subsection{Motiva��o}

\begin{frame}
	\frametitle{Motiva��o}
	\pause
	  \begin{itemize}
		\item Great Firewall of China
		\pause
		\item Tor
		\pause
		\item Cypherpunk
		\pause
		\item John Perry Barlow
	  \end{itemize}
\end{frame}

\section{Fundamentos}
\subsection{Fundamentos}
\begin{frame}
	\frametitle{Fundamentos}
	\pause
	  \begin{itemize}
		\item O Protocolo IP
		  \begin{itemize}
			\item Conte�do exposto 
		  \end{itemize}
		\pause
		\item Criptografia
		\pause
		\item Servidores DNS
		\pause
		\item ISPs e suas rela��es
		  \begin{itemize}
			\item Peering
			\pause
			\item Tr�nsito (\$)
			\pause
			\item Ponto de Troca de Tr�fego (PTT) ou \emph{Internet Exchanging Point} (IXP)
		  \end{itemize}
	  \end{itemize}
\end{frame}

\section{Censura na Internet}
\begin{frame}
Censura na Internet
\end{frame}


\subsection{Remo��o de conte�do}
\begin{frame}
	\frametitle{Remo��o de conte�do}
\end{frame}

\begin{frame}
	\frametitle{Terceiriza��o da censura}
\end{frame}

\begin{frame}
	\frametitle{Legitimidade da censura}
\end{frame}

\subsection{Bloqueio de acesso a conte�do}
\begin{frame}
	\frametitle{DNS Poisoning}
\end{frame}
\begin{frame}
	\frametitle{Filtragem de pacotes: Bloqueio a IPs}
\end{frame}
\begin{frame}
	\frametitle{Filtragem de pacotes: Inspe��o autom�tica de conte�do}
\end{frame}


\end{document}
